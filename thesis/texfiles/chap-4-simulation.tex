%Chapter 4
\chapter{Showcase of Example Unit and Simulation}
\label{chap:simulation}
%
\section{Introduction to the Simulation Environment}

\section{Introduction to the developed models}

\subsection{Plant Model}
\subsubsection{Evapotranspiration}
References for the ET calculation:\\
https://etcalc.hydrotools.tech/pageMain.php\\
https://www.fao.org/4/X0490E/x0490e07.htm\\
https://www.fao.org/4/X0490E/x0490e0k.htm\\

\subsubsection{Yield}
An initial state had to be given for $x_{nsdw}$ and $x_{sdw}$ to avoid a division with zero.

\subsection{Physical Environment Model}

\subsection{LED Model}
\subsection{Pump Model}
\subsection{Air Conditioning Model}
Note that $\text{CO}_2$ concentrations are not dynamically calculated in the simulation and the humidity contribution from the plants is not considered.
The reason for this, is that the air volume component allowing for dynamic calculation led to frequent convergence errors not further investigated in this work.
The chosen value for $\text{CO}_2$ is 365 ppm, which is an average value for the atmosphere https://doi.org/10.1111/j.1365-3040.2007.01641.x.
For the \ac{vpd} calculation, humidity levels are taken from weather data and temperature from the farm air volume.



\section{Simulation Architecture}
\textcolor{blue}{Buildings.ThermalZones.ReducedOrder.RC.TwoElements for Radiation modelling}

\section{Analysis of Energy Use and Comparison with State of the Art}
\label{sec:sim-energy-and-comparison}

% \section{Results}
% \label{sec:simulation-results}
