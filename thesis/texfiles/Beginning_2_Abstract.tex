\chapter{Abstract}
%
This chapter is under construction.\\
This work will introduce a plant irrigation system in the form of panels.
These panels shall be mounted on building facades and be protected from the elements by an additional layer of glass.
With this we can provide all of the benefits over traditional agriculure which have been discussed before.
Simultaneously this arrangement addresses the main problem of present vertical farming systems by not relying on a completely artificial environment and instead using existing resources to cultivate the plants.
Namely natural lighting by the sun and vertical area of city infrastructure.
% This provides the benefits in contrast to traditional agriculture clean, regional food is produced.
% However in contrast to commercially operating vertical farms this arrangement allows to use existing resources to cultivate the plants.
% However it still comes with all of the benefits over traditional agriculture which have been discussed before.
% as the area use is vertical instead of horizontally and water use is cut significantly.

Additionally it provides even more benefits resulting from the tight integration into its environment and distributed nature of deployment.
double use as building insulation.

This work will introduce a urban farming concept providing clean, regional food while simultaniously providing insulation to existing buildings and improving city climate.
The solution presented consists of panels which can be retrofitted on existing building
Let us imagine a future city where old buildings have been retrofitted with insulating tiles. These tiles shall 
- improvement of quality of life factors inside cities such as improved air quality, beautifying building facades and creating awareness for plants and human food production
- providing clean, regional food for cities
- insulate existing buildings for more energy efficiency and sound isolation
- help with regulating city climate during heat waves
