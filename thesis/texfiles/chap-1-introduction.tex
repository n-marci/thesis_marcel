\chapter{Introduction}
\label{chap:Introduction}

Lorem ipsum ...
This is some sample text.\\
This is more sample text :).
And I have even more sample text.
Test.
%
\begin{figure}[H]
	\centering
	\includegraphics[width=1\textwidth]{img/FAU}
	\caption{Image caption}
	\label{Image_Label}
\end{figure}
%
\noindent
%
Lorem ipsum ...

\section{Motivation}
\label{sec:Motivation}
In the last two centuries human civilization has seen tremendous growth, a rise in global interconnectedness and urbanization.
These trends are posed to continue at a rapid rate and provide mankind with prosperity never imaginable to our ancestors.
Unfortunately as everything in life these developments also come with significant drawbacks we as a society need to address.

One, interconnectedness comes at the cost of reliance.
The division of labor on a global scale has produced the curious situation where some nations are not able to provide food for their own people. \textcolor{blue}{needs ref.}
An arrangement which previously has taken down not only nations but entire civilizations.
Something as basic as food supply should be the upmost priority for a government serving its people.
However now, agricultural highly productive nations such as Ukraine are exporting much of their produce, providing a stable food supply to the world.
But with this we can see multiple problems.
On the one hand climate conditions and regions may shift significantly in the future which would result in previously very productive areas becoming less fertile.
On the other hand in a global economy every nation is its own actor.
Recent history has revealed that overdependence on an entity you have no control over or which safety you can not insure, can lead to catastrophic consequences.
But not only autocracies with their agenda pose a risk to this configuration.

Second, higher prosperity has led to higher resource usage and an exploitation of our environment.
Especially cities are drivers of incredible economic prosperty as they provide a dense and efficient network of people, equipment and services.
What has traditionally be lacking in these urban environments however is food production.
And so humans still need to rely on area and water intensive traditional agriculture in rural regions to produce their food.

% This work presents a concept which aims to address these problems \- \ac{cea} and in particular urban vertical farming.



Higher interconnectedness comes at the cost of increased reliance on other global actors and autocrats.
And as we have seen in the last years with the attack on Ukraine, heavy reliance on non democratic regimes 
food safety is still an issue in the world - famines have played significant role in downfall of civilizations in the past.

The motivation for this work is threefold.
One, crafting ideal human environment in the context of smart city.
- Taking advantage of synergies.
Two, resilience and national autonomy of food supply.
- Preparedness for shifting climate.
Three, most important to the author, global sustainability of human civilization.
- water use - pesticide use and water ?artrification? - local food production - freeing up area for natural ecosystems

?Mehr auslegen einer Vision? - Imagine the future city. Clean green walls dampen the sound of cars, provide cooling in the hot summer months. %@vincent
More people in cities.
More sustainable production systems needed.
Land use of agriculture precise value.
Water use of agriculture precice value.
Greenhouse Gas emissions of agriculture precise value.
Good sources in this paper https://doi.org/10.3390/horticulturae10020117.



\section{Problem Statement}
The last decade has seen the advent of a novel approach to agriculture to tackle these problems.
\ac{cea} and in particular urban vertical farming want to control the plants environment more fully to lessen the reliance on climate.
This method has a number of benefits over traditional agriculture.
- less area needed
- significantly less water use
- no water ?atrification?
- no dependence on climate conditions
- no need for pesticides and therefore clean food
- more regional food production
If we want to tackle the problems posed above
Introduce benefits of vertical farming

Companies such as ... and ... try to seperate the plants completely from the elements and control the environment they are in fully.
This of course is great for reproducibility and quality.
However as we will show later in chapter \ref{chap:analysis-and-arch} current commercially operating farms with this approch have one main problem.
Energy consumption.
This makes them economically less competitive to traditional agriculture and shifts the resource usage from water and land area to energy.
Not ideal for Germany, a country which still relies to ... \% on fossil fuels for its energy production \textcolor{blue}{ref for concrete number}.

From the three main motivations laid out before we can extract the main problems of these areas.
One - Air quality - disconnection from nature - buildings not properly insulated, taking advantage of synergies. 
Two - Reliance on a few agricultural plentiful areas in the world to sustain the hunger of the human population - These might change as climate patterns will shift.
Three - 
High resource use of cities.
Uncertainty of future climate.
This section laid out a number of different problems. This work can not adress all, but tries to connect 

% \subsection{State of the Art}
Let us first examine current \ac{cea} and vertical farming approaches to get a better understanding of the solution methods and shortcomings.
Current solutions try to achieve high degree of automation and full control over the environment.
This results in high energy



\section{Solution Proposal}
% \subsection{Goals and Contributions}
This work will introduce a plant irrigation system in the form of panels.
These panels shall be mounted on building facades and be protected from the elements by an additional layer of glass.
With this we can provide all of the benefits over traditional agriculure which have been discussed before.
Simultaneously this arrangement addresses the main problem of present vertical farming systems by not relying on a completely artificial environment and instead using existing resources to cultivate the plants.
Namely natural lighting by the sun and vertical area of city infrastructure.
% This provides the benefits in contrast to traditional agriculture clean, regional food is produced.
% However in contrast to commercially operating vertical farms this arrangement allows to use existing resources to cultivate the plants.
% However it still comes with all of the benefits over traditional agriculture which have been discussed before.
% as the area use is vertical instead of horizontally and water use is cut significantly.

Additionally it provides even more benefits resulting from the tight integration into its environment and distributed nature of deployment.
double use as building insulation.

This work will introduce a urban farming concept providing clean, regional food while simultaniously providing insulation to existing buildings and improving city climate.
The solution presented consists of panels which can be retrofitted on existing building
Let us imagine a future city where old buildings have been retrofitted with insulating tiles. These tiles shall 
- improvement of quality of life factors inside cities such as improved air quality, beautifying building facades and creating awareness for plants and human food production
- providing clean, regional food for cities
- insulate existing buildings for more energy efficiency and sound isolation
- help with regulating city climate during heat waves
