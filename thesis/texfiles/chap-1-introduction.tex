\chapter{Introduction}
\label{chap:Introduction}

Lorem ipsum ...
This is some sample text.\\
This is more sample text :).
And I have even more sample text.
Test.
%
\begin{figure}[H]
	\centering
	\includegraphics[width=1\textwidth]{img/FAU}
	\caption{Image caption}
	\label{Image_Label}
\end{figure}
%
\noindent
%
Lorem ipsum ...

\section{Motivation}
\label{sec:Motivation}
The motivation for this work is threefold.
One, crafting ideal human environment in the context of smart city.
- Taking advantage of synergies.
Two, resilience and national autonomy of food supply.
- Preparedness for shifting climate.
Three, most important to the author, global sustainability of human civilization.
- water use - pesticide use and water ?artrification? - local food production - freeing up area for natural ecosystems

?Mehr auslegen einer Vision? - Imagine the future city. Clean green walls dampen the sound of cars, provide cooling in the hot summer months. %@vincent



\section{Problem Statement}
From the three main motivations laid out before we can extract the main problems of these areas.
One - Air quality - disconnection from nature - buildings not properly insulated, taking advantage of synergies. 
Two - Reliance on a few agricultural plentiful areas in the world to sustain the hunger of the human population - These might change as climate patterns will shift.
Three - 
High resource use of cities.
Uncertainty of future climate.
This section laid out a number of different problems. This work can not adress all, but tries to connect 

% \subsection{State of the Art}
Let us first examine current \ac{cea} and vertical farming approaches to get a better understanding of the solution methods and shortcomings.
Current solutions try to achieve high degree of automation and full control over the environment.
This results in high energy



\section{Solution Proposal}
% \subsection{Goals and Contributions}
First we shall look
- Gliederung der Arbeit

 \begin{tikzpicture}                                                                                                 
 \draw[->] (0,0) -- (4,0) node[right] {East};                                                                        
 \draw[->] (0,0) -- (0,4) node[above] {North};                                                                       
 \draw (2,0) -- (2,3) node[midway, right] {GHI};                                                                     
 \end{tikzpicture}       

                                                                                                                     
 \begin{tikzpicture}                                                                                                 
 \draw[->] (0,0) -- (4,0) node[right] {East};                                                                        
 \draw[->] (0,0) -- (0,4) node[above] {North};                                                                       
 \draw (2,0) -- (2,3) node[midway, right] {DNI};                                                                     
 \draw[->, red] (2,3) -- (2,0);                                                                                      
 \end{tikzpicture}  

 \begin{tikzpicture}                                                                                                 
 \draw[->] (0,0) -- (4,0) node[right] {East};                                                                        
 \draw[->] (0,0) -- (0,4) node[above] {North};                                                                       
 \draw (2,0) -- (2,3) node[midway, right] {DHI};                                                                     
 \draw[->, blue] (2,3) -- (1,0);                                                                                     
 \draw[->, blue] (2,3) -- (3,0);                                                                                     
 \end{tikzpicture}                                                                                                   

                                                                                                                      
 \begin{tikzpicture}                                                                                                 
 \draw[->] (0,0) -- (4,0) node[right] {East};                                                                        
 \draw[->] (0,0) -- (0,4) node[above] {North};                                                                       
 \draw (2,0) -- (2,3) node[midway, right] {POA};                                                                     
 \draw[->, red] (2,3) -- (2,0);                                                                                      
 \draw[->, blue] (2,3) -- (1,0);                                                                                     
 \draw[->, blue] (2,3) -- (3,0);                                                                                     
 \draw[->, green] (2,0) -- (2,1);                                                                                    
 \end{tikzpicture}      
