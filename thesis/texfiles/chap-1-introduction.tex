\chapter{Introduction}
\label{chap:Introduction}

% %
% \begin{figure}[H]
% 	\centering
% 	\includegraphics[width=1\textwidth]{img/FAU}
% 	\caption{Image caption}
% 	\label{Image_Label}
% \end{figure}
% %
% \noindent
% %
% Lorem ipsum ...

\paragraph{Motivation}
\label{par:Motivation}
In the last two centuries human civilization has seen tremendous growth, a rise in global interconnectedness and urbanization.
These trends are posed to continue at a rapid rate and provide mankind with prosperity never imaginable to our ancestors.
Unfortunately as everything in life these developments also come with significant drawbacks, we as a society need to address.

One, interconnectedness comes at the cost of reliance.
The division of labor on a global scale has produced the curious situation where some nations are not able to provide food for their own people \textcolor{Blue}{needs ref.}
An arrangement which previously has taken down not only nations but entire civilizations \textcolor{Blue}{needs ref}.
Something as basic as food supply should be the upmost priority for a government serving its people.
However now, agricultural highly productive nations such as Ukraine are exporting much of their produce, providing a stable food supply to the world.
But with this we can see two major problems.
On the one hand, recent history has blatantly revealed that we live on a global stage with many different actors and their own agendas.
Relying too heavily on entities a nation can not control or for which their safety can not be insured, poses a serious concern for said nation.
On the other hand, human made carbon emissions will have impacts on the climate which can not be predicted fully.
% What is certain however is that there will be changing climate and therefore weather patterns in many regions of the world.
It is certain however, that current climate and weather patterns will shift in many regions of the world.
The stability of these systems constitutes a big factor in what makes highly fertile lands the 'breadbaskets of earth'.
This dependability can not be relied upon in the future.
% Changing conditions will result in much heavier intrusions into the natural growth cycle, quickening depletion of soil quality and water reserves.
% This of couse will result in considerable less productivity of these highly important regions for the global food supply chain.

But not only security of the food supply is a concern.
Humanities resource usage and exploitation of the environment has spelled doom for biodiversity on planet earth.
% \textcolor{Blue}{concrete number for biodiversity loss?}.
Part of the reason for this huge impact is traceable to our civilizations' land use.
% Agriculture is one of the major contributers to civilizations land usage.
Approximately ...\% of the earths land surface is occupied by agriculture \textcolor{Blue}{needs ref}.
A startling fact considering that the vast majority of people live in cities, which themselves are highly space efficient.
Even owing much of their success to the tight integration of people, services and industry.
On the flip side these urban spaces will get less livable in the future.
They are mostly comprised of concrete, asphalt and glass, trapping much of the incoming heat.
This stands in stark contrast to rural areas in which natural vegetation provides evaporative cooling and shade.
Current technological cooling solutions are energy intensive and constitute only a remedy for the symptom, not fixing the underlying cause.

% In recent years a few approaches have been tried to tackle these issues.

% Urban greening is probably best done with parks and diverse trees scattered across the city.
% These not only provide a better city climate, but can also be used as recreational spaces for the citizens.
% However most of the space in a city is already occupied by buildings \textcolor{Blue}{needs ref} where a remodelling is not feasible.
% This creates the need for more integrated approaches to urban greening.
% Rooftop gardens and Green Facades are an obvious method to relief the problems.

These issues are getting addressed slowly and separately for now.
To tackle heat buildup in cities, urban greening can be used.
This comes in the form of public parks, grassy areas for recreational use and trees to provide shade.
But since a lot of city area is already occupied by buildings, this is no solution everywhere.
Rooftop gardens and facade greening are a logical next step to increase urban plant density.
And indeed indoor temperatures and air quality around buildings following this approach are measurably improved \textcolor{Blue}{needs ref}.

Next, to lessen the reliance and impact of food supply on the climate, \ac{cea} and in particular Urban Vertical Farming aim to control the plants' environment fully.
This enables traditionally less arable regions to take food production into their own hands and grants a number of other benefits. %over traditional agriculture.
By virtue of growing vertically and optimizing the plant environment, area use is significantly reduced.
 % the area requirement for the same amount of produce is significanty reduced. %by ... \textcolor{Blue}{needs ref}.
Need for fresh water is cut to only 5 to 10 \% of traditional systems \textcolor{Blue}{needs ref}. %, depending on the irrigation technology utilized.
And because the plants are entirely kept inside their own artificial ecosystem, pests and therefore pesticides are of no concern.
Allowing clean food production transcending even organic standards.
Fertilizer can be kept inside this microcosm as well and does not seep into the soil, reducing freshwater eutrophication.
Lastly these farms can be deployed wherever there is energy and water infrastructure.
% enough energy and water supply to care for the plants.
This enables to grow food far more regional than possible at the current moment.
% - less area needed
% - significantly less water use
% - no water ?atrification?
% - no dependence on climate conditions
% - no need for pesticides and therefore clean food
% - more regional food production
% Vertical farming tries to address food security and sustainability while urban greening deals with the city climate issues.
% In recent years a new frontier in agriculture is trying to bring food production into the cities.

Albeit these promising qualities, a green city revolution has failed to materialize so far.
Since nature is messy and changes over time, facade greening constitutes an additional burden, without providing a tangible advantage to the building owner.
% Plants need to be cut at regular intervals and inspections of the building structure integrity have 
Maintenance in the form of cutting plants and inspecting the integrity of building structure becomes necessary.
% example of green tower in italy.
% For Urban Greening
% This work will not illuminate
This work will not illuminate these issues in detail but offers inherent relief by greening with crop plants.
% These shortcomings will not be looked at in detail with this work but the relief comes naturally from greening with crop plants.
They provide economic value and already presuppose a controlled environment for the plants to grow in.

% This is due to a number of obstacles.
% There are two main obstacles identified by the author, which hold back the adoption of Vertical Farmin.
There are two main obstacles which hinder adoption of vertical farming as identified by this study.
One: The types of plants which can be grown is limited.
Especially when taking profitability into consideration.
Mostly leafy greens and microgreens are cultivated to date. %economically in this artificial environment to date.
Two: The energy consumption is significantly higher than traditional agriculture @barbosa2015.
This makes these farms less competitive and shifts resource demand from water and land area to energy.
In countries like Germany, where fossil fuels still comprise a significant part of the energy production, this is a notable concern.
% Not ideal for example in Germany, a country which still relies to ... \% on fossil fuels for its energy production \textcolor{Blue}{needs ref}.

% vertical farming problems:
% - small range of plants which can be cultivated in this highly artificial environment.
% - energy use orders of magnitude higher than traditional agriculture @barbosa2015.

% urban greening problems:
% - additional financial burden to maintain with no direct advantage to the building owner
% - nature is 'messy' and hard to deal with

% This work will focus on adressing the shortcomings of vertical farming.
% But by nature of its deployment we can also harvest some of the advantages of urban greening.

% On the one hand climate conditions and regions may shift significantly in the future which would result in previously very productive areas becoming less fertile.
% On the other hand in a global economy every nation is its own actor.
% Recent history has revealed that overdependence on an entity you have no control over or which safety you can not insure, can lead to catastrophic consequences.
% But not only autocracies with their agenda pose a risk to this configuration.

% Second, higher prosperity has led to higher resource usage and an exploitation of our environment.
% Especially cities are drivers of incredible economic prosperty as they provide a dense and efficient network of people, equipment and services.
% What has traditionally be lacking in these urban environments however is food production.
% And so humans still need to rely on area and water intensive traditional agriculture in rural regions to produce their food.

% This work presents a concept which aims to address these problems \- \ac{cea} and in particular urban vertical farming.



% The motivation for this work is threefold.
% One, crafting ideal human environment in the context of smart city.
% - Taking advantage of synergies.
% Two, resilience and national autonomy of food supply.
% - Preparedness for shifting climate.
% Three, most important to the author, global sustainability of human civilization.
% - water use - pesticide use and water ?artrification? - local food production - freeing up area for natural ecosystems

% From the three main motivations laid out before we can extract the main problems of these areas.
% One - Air quality - disconnection from nature - buildings not properly insulated, taking advantage of synergies. 
% Two - Reliance on a few agricultural plentiful areas in the world to sustain the hunger of the human population - These might change as climate patterns will shift.
% Three - 
% High resource use of cities.
% Uncertainty of future climate.
% This section laid out a number of different problems. This work can not adress all, but tries to connect 

% ?Mehr auslegen einer Vision? - Imagine the future city. Clean green walls dampen the sound of cars, provide cooling in the hot summer months. %@vincent
% More people in cities.
% More sustainable production systems needed.
% Land use of agriculture precise value.
% Water use of agriculture precice value.
% Greenhouse Gas emissions of agriculture precise value.
% Good sources in this paper https://doi.org/10.3390/horticulturae10020117.

% \textcolor{Blue}{Nochmal in Obsidian checken für Notizen zu optimaler Introduction}
% \textcolor{Blue}{Unterteilen in einzelne Sections oder monolithisch?}

% Now imagine the future smart city.
% In the view of the author this should be the optimal environment hand crafted to the needs of its inhabitants.
% Smart, clean infrastructure serves the people instead of polluting their health and environment.

\paragraph{Procedure}
% Mitigating the flaw of excessive energy demand will be the main focus of this work.
% By this the natural illumination of the sun is put to use and energy demand for artificial lighting is cut significantly.
% From this starting point we 
The main objective of this work is to drastically reduce the energy requirements of Vertical Farming while maintaining a semi-controlled environment for the crops to grow in.
Especially optimal lighting conditions for the plants shall be maintained for reasons which are discussed in Section \ref{sec:system-analysis}.
% since illumination accounts for the highest power draw.
% This is shown later.

To accomplish this goal, first the \nameref{chap:fundamentals} introduce some basic concepts and terminology employed in this work.
We will then look at existing commercial and academic vertical farming systems and analyze strengths and deficiencies.
As reasoned later in the \nameref{chap:analysis-and-arch}, the main issue holding back adoption are high energy usage requirements.
% The conclusions of this inquiry are 
This is the basis of the novel concept presented.
To minimize energy consumption, natural light shall be used.
This is accomplished by retrofitting building facades with the proposed system.
This choice directly results in an obvious synergy.
% Creating an outer shell around the building envelope, to insulate it.
Using the vertical farming infrastructure as an outer layer to insulate existing buildings or new architectural projects.
To the best of the authors' knowledge, a system like this has not been suggested so far.
Integrating vertical farming with building climate control has been proposed before \textcolor{Blue}{needs ref}.
However, this paper suggests using the basement for farming.
A space which is currently already in use for most buildings.
% Requirements to judge feasibility of the concept are stipulated in section \ref{sec:feasibility}.
Coming back to this work, in section \ref{sub:stru-arch} the general architecture of the system is constructed and visualized with SysML diagrams.
The concept of the plant panel is introduced.
This presents an idea on which much of the concept hinges.
The vision of the architecture is shown via Blender models representing a tangible implementation at Friedrich-Alexander-University.
A more concrete power system is developed in section \ref{sub:power-arch} and components are picked to actualize the aforementioned realization.
This will later guide the models developed for the simulation.
Section \ref{sec:feasibility} stipulates requirements to judge feasibility of the present retrofit concept.
Also, metrics to evaluate these requirements are discussed.
% In these the 'Elektrotechnikturm', location of the Chair of Electrical Smart City is transformed.
Chapter \ref{chap:simulation} then implements this example unit in a Modelica simulation.
Originating from the feasibility requirements and plant needs, the general simulation architecture is built up.
% We chose lettuce as the crop plant.
Mathematical models describing water use and yield output for the crop are implemented and combined as a Modelica model.
Building on the work of the Modelica Buildings Library, an investigation into the thermal and energy balances is set up.
A simulation of the physical environment is constructed and interactions with the engineered system are taken into account.
% Interactions with the control system are laid out
Section \ref{sec:sim-energy-and-comparison} compares the resulting energy requirements to current vertical farming systems.
A suitable scale for a solar installation will be given and the feasibility assessed.
The results are presented in chapter \ref{chap:results}.
Further evaluation and resulting conclusions are discussed in chapter \ref{chap:discussion}.
In chapter \ref{chap:conclusion} the findings are summarized and areas of further interest are laid out.

% So far modelica libraries only implement tomato model
% The interfaces with the different adjacent systems will be made clear.

% This is accomplished by utilizing building facades as agricultural area.

% \textcolor{Blue}{Ok to separate into two sections or better alles aus einem Guss?}
