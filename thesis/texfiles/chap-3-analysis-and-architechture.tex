%Chapter 3
\chapter{Theoretical Analysis and Architectural Approach}
\label{chap:analysis-and-arch}
%
\section{Energy Analysis}
\label{sec:energy-analysis}
Let us first examine current \ac{cea} and vertical farming approaches to get a better understanding of the solution methods and shortcomings.
Current solutions try to achieve high degree of automation and full control over the environment.
This results in high energy

Companies such as ... and ... try to seperate the plants completely from the elements and control the environment they are in fully.
This of course is great for reproducibility and quality.
However as we will show later in chapter \ref{chap:analysis-and-arch} current commercially operating farms with this approch have one main problem.
Energy consumption.
This makes them economically less competitive to traditional agriculture and shifts the resource usage from water and land area to energy.
Not ideal for Germany, a country which still relies to ... \% on fossil fuels for its energy production \textcolor{blue}{needs ref}.



\section{Presentation of the General Concept}
\label{sec:concept}
\section{Feasibility}
\label{sec:feasibility}
Metrics to evalute feasibility of the concept:
\begin{itemize}
	\item The energy consumption can be met through a solar installation covering at most the area on the roof.
	\item Yield can offset investment costs in a reasonable timeframe.
	\item Farm provides measurable insulation increase in comparison with the 'naked' building.
	\item Acceptance of potential customers to put a greenhouse on the side of their buildings (not evaluated in this work)
\end{itemize}

\section{Energy System Architecture}
\label{sec:architecture}

\subsection{Choice of Components}
