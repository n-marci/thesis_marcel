%Chapter 2
\chapter{Fundamentals}
\label{chap:fundamentals}

\section{Heat Transfer}
\subsection{Types of Heat}
Heat can be classified into two different forms.
There is sensible heat which directly causes a temperature change in a material.
And there is latent heat which is responsible for the phase change of a material.
During the phase change, there is no temperature change from heat added into or substracted from the system.
Total heat transfered during a process is denoted by $Q$ and the rate at which this happens is signified with $\dot{Q}$ carrying the unit Watt.
This heat transfer rate $\dot{Q}$ is what we will look at next.

\subsection{Types of Heat Transfer}
@cengel2003
Heat transfer can fundamentally occur in three different forms.
Conduction, Convection and Radiation.

\textit{Conduction} refers to heat moving through a material.
It is characterised by the heat conductivity $k$ specific to the substance in question and can be modeled by Fourier's law
$$
\dot{Q}_{cond} = -k \frac{A}{L} \Delta T
$$,
where $A$ is the area through which the conduction takes place, $L$ is the distance and $\Delta T$ is the temperature difference.
\textit{Convection} is heat transfered on the boundary between a solid and a fluid.
The characteristic value for this interaction is the convection heat transfer coefficient $h$ while the mathematical description is given by Newton's law of cooling
$$
\dot{Q}_{conv} = h A \Delta T
$$,
with $A$ being again the area, and $\Delta T$ the temperature difference.
\textit{Radiation} 

To be able to understand the implementation of the simulation discussed later in \textcolor{blue}{needs ref}, we first need to introduce the basics of heat transfer. 
There are three different forms in which heat transfer can occur.
First Conduction, which describes heat moving through a solid material. This property can be characterised by the heat conductivity $k$ for the specific material.
The mathematical model equation will be introduced later with the heat capacity in \ref{sub:ther-props}.
Second we have Convection.
This is heat transfer which happens between the surface of a solid material and a fluid.
Last the transfer can occur via radiation.
As the name implies this is basically heat energy transmitted with electromagnetic rediation.
These will all be modeled seperately later.
\textcolor{blue}{How about heat transfer via mass transfer?}
\textcolor{blue}{some sources also have advection - what's that?}
\textcolor{blue}{Difference between sensible and latent heat flow.}

\subsection{Other important thermodynamic properties}
\label{sub:ther-props}
Explain Heat Capacity.



\section{Systems Modelling Approach}

\section{Agricultural and CEA Basics}
In the center of \ac{cea} stands the plant.
The environment is crafted to provide optimal conditions.

\subsection{Irradiance}
\subsubsection{Irradiance on a tilted surface}

\subsubsection{Solar spectrum and photosynthesis}
When assessing the optimal lighting conditions for plant growth, several factors need to be illuminated.
Pun intended.
Light spectrum, Instantenous light intensity, Cumulative light amount and Photoperiod

For quantifying \textit{spectrum} and \textit{instantenous intensity} we will introduce \ac{par} and \ac{ppfd}.
As discussed before we want to 
How do we quantify the solar irradiance 
Plants use solar radiation in the spectrum from 400 nm to 700 nm \textcolor{blue}{needs ref} for photosynthesis.
This is only a portion of the actual solar radiation which is hitting earth.
For the \textit{natural} solar radiation in the context of plant growth, the concept of \ac{par} is most often used.
This describes the solar radiation which lies inside the aforementioned range for photosynthesis and can be calculated with a simple conversion factor @reis2020.
% To convert the irradiance on a tilted surface we calculated before to the \ac{par}, there exists a simple conversion factor @reis2020.
Meanwhile when using \textit{artificial} lighting, \ac{ppfd} is used to describe the relevant radiation.
\ac{par} and \ac{ppfd} quantify \textit{light spectrum} and \textit{instantenous intensity}.
They both carry the same unit and except for their natural or artificial origin, can be treated the same.

To quantify the \textit{cumulative light amount} and \textit{photoperiod} for a whole day, we simply accumulate \ac{par} and \ac{ppfd} over one day.
This is called \ac{dli}.

LEDs are chosen because of their high efficiency and possibility to adjust the light spectrum granuarly.

\subsection{Irrigation}
Soilless Agriculture - Hydroponics - Aeroponics

Aeroponics and specially fogponic system is chosen because of the lightweight nature and ease of deployment.

\subsection{Atmosphere}
Vapor pressure - VPD
CO2 concentrations


