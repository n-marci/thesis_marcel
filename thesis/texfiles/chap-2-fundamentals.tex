%Chapter 2
\chapter{Fundamentals}
\label{chap:fundamentals}

\section{Thermodynamics}
\textcolor{Blue}{How do I cite best here? Whole section is based on} @cengel2003.
This work wants to model plant growth and insulation potential.
These statements require temperature information.
% Naturally we need temperature information to achieve that.
Therefore, this section introduces some fundamentals from thermodynamics.
They will make it possible to simulate heat flows and gather necessary data.

% As this work wants to judge plant growth, temper

% In this work the physical environment of the engineered system is simulated.
% The resulting temperature information is used in the further modelling of plant growth.
% To do this, we need to introduce some concepts from thermodynamics.
% To model the physical environment our engineered system will be placed in, we need to introduce some concepts to describe heat flows.
% This will enable us to later judge insulation potential.
% Additionally, the resulting temperature information is needed as an input for the plant model.

\subsection{Types of Heat}
Heat can be classified into two different forms.
There is sensible heat which directly causes a temperature change in a material.
And there is latent heat which is responsible for the phase change of a material.
During the phase change, there is no temperature change from heat added into or subtracted from the system.
% Water evaporating from a leave falls into this category.
Total heat transferred during a process is denoted by $Q$ and the rate at which this happens is signified with $\dot{Q}$ carrying the unit Watt.
This heat transfer rate $\dot{Q}$ is what we will look at next.

\subsection{Heat Transfer}
\label{sub:heat-transfer}
Heat transfer can fundamentally occur in three different forms.
Conduction, Convection and Radiation.

\paragraph{Conduction} This refers to heat moving through a material.
It is characterized by the heat conductivity $k$ (\si{\W\per\m\per\K}) specific to the substance in question and can be modeled by Fourier's law
$$
\dot{Q}_{cond} = -k \frac{A}{L} \Delta T
$$
where $A$ (\si{\square\m}) is the area through which the conduction takes place, $L$ (\si{\m}) is the distance and $\Delta T$ (\si{\K}) is the temperature difference.

\paragraph{Convection} This is heat transferred on the boundary between a solid and a fluid.
The characteristic value for this interaction is the convection heat transfer coefficient $h$ (\si{\W\per\square\m\per\K}) while the mathematical description is given by Newton's law of cooling
$$
\dot{Q}_{conv} = h A \Delta T
$$
with $A$ being again the area, and $\Delta T$ the temperature difference.

\paragraph{Radiation} This describes heat transfer via electromagnetic waves.
This property can be emitted or absorbed.
Any material possessing a temperature greater absolute zero will emit some heat to its surroundings.
For a black body -- an idealized concept absorbing all incident radiation -- this heat flux density is given by the Stefan-Boltzmann Law.
For real materials the emissivity $\epsilon$ (-) and the objects' surface area $A$ are taken into account to get
$$
\dot{Q}_{rad} = \epsilon \sigma A (T^4 - T_{surr}^4)
$$
where $T$ (\si{\K}) is the material temperature, $\sigma$ (\si{\W\per\square\m\per\K\tothe{4}}) is the Stefan-Boltzmann constant and $T_{surr}$ (\si{\K}) describes the temperature of an idealized sphere infinitely far from the object.
When taking about incoming radiation, we have the characteristic value of absorptivity $\alpha$ (-).
This is combined with the incident radiation $\dot{Q}_{inci}$ (\si{\W}) to obtain 
$$
\dot{Q}_{abso} = \alpha \dot{Q}_{\text{inci}} 
$$
for captured heat flux by a material.

% Conveniently these physical relations are already modeled in the Modelica Standard Library and further built upon with the Buildings Library.
% \textcolor{Blue}{Should I introduce here already the modelica models which implement this?}



\subsection{Other relevant thermodynamic properties}
\label{sub:ther-props}
Heat Capacity.\\
Heat transfer via mass transfer.\\
\textcolor{Blue}{Shall I put references here for what the fundamentals are needed? Like latent heat to determine evaporation cooling, mass transfer for ventilating?}



% \section{Systems Modelling Approach}

\section{Agricultural and CEA Basics}
\label{sec:fund-cea}
As the name suggests, \acl{cea} is about controlling the environment of a crop.
So what needs to be managed to provide a habitat for these green organisms?
% So which parts of the normal habitat need to be managed?
This can most easily be answered by abstracting the plant down to the process of photosynthesis.
$$
6 \text{CO}_2 + 6 \text{H}_2\text{O} \rightarrow \text{C}_6\text{H}_{12}\text{O}_6 + 6 \text{O}_2
$$
The output we usually care about when growing crops is glucose $\text{C}_6\text{H}_{12}\text{O}_6$ as well as more complex carbohydrates built up from this constituent.
After water, these make up the bulk of the mass of the plant.
This is the reason to optimize this process.
As we can see, there are two chemical inputs necessary.
Water is taken up by the roots, so \textit{Irrigation} is required.
Carbon Dioxide diffuses into the leaves from the surrounding air and so the plant benefits from a controlled \textit{Atmosphere}.
Additionally, there is another input not directly apparent in the equation.
% Not directly apparent in the equation, there is an additional input.
Since photosynthesis is an endothermic reaction, energy needs to be supplied.
This is done with \textit{Illumination}.
These are the three domains on which technological solutions hinge to provide optimal control.

% There can be more nuance when diseases and plant health come into play.

% To understand \ac{cea}, we need to understand photosynthesis.

% This will help us to understand what technology is needed in a modern agricultural system.
% And how this influences the biological system we are trying to take care of.
% This is important as we are trying to gauge the yield output of the crop later.

% In the center of \ac{cea} stands the plant.
% The environment is crafted to provide optimal conditions.

\subsection{Illumination}
For illumination two foundational notions have an influence on the plant.
Instant irradiance and the characteristics on how this instantaneous value is supplied over time.
% These are discussed in the following section.

% \textit{Instant Irradiance} can be further divided into light spectrum and intensity.
% The aggregation of this instantaneous value is what is captured by time dependent factors.
% Classify illumination broadly into two 
% For illumination a few factors are of importance.
% Instant irradiance and time dependent factors like dli and dark period.

% For irradiance there are two important factors which define instant irradiance.
% Spectrum and irradiance power.
\subsubsection{Instant Irradiance}
The instantaneous radiation can further be divided into light spectrum and intensity for our application.
% Also natural and artificial sources are examined separately.
The two concepts \ac{par} and \ac{ppfd} quantify these qualities for natural and artificial sources respectively.
They both carry the same unit (\si{\umol\per\square\m\per\s}) and except for their origin, can be treated the same.
As their name implies these values characterize the radiation which can be used in photosynthesis.
% For both quantifying spectrum and intensity we will introduce \ac{par} and \ac{ppfd}.
% There are two determinants influencing the effect of instantaneous radiation on plant life.
% Light intensity and spectrum.
\paragraph{Light Spectrum}
% \subsubsection{Irradiance on a tilted surface}
% \subsubsection{Solar spectrum and photosynthesis}
Plants use light in the spectrum from 400 nm to 700 nm \textcolor{Blue}{needs ref}.
This is only a portion of the natural solar spectrum which is hitting earth.
Luckily a simple conversion factor exists to convert the suns' radiation usually provided in \si{\W\per\square\m} to \ac{par} @reis2020.
% For the \textit{natural} solar radiation in the context of plant growth, the concept of \ac{par} is most often used.
% This describes the solar radiation which lies inside the aforementioned range for photosynthesis and can be calculated with a simple conversion factor @reis2020.

For artificial radiation you are able to choose a light source with specific spectrum characteristics.
This used to be more constrained in the past, but the proliferation of \acsp{led} allows for fine-grained adjustments of light quality.
% Sources with all their spectrum inside the photosynthesis range are possible.
Sources catering their spectrum specifically to the photosynthetic range are widely available.
And so we can take the light output right as \ac{ppfd}.

\paragraph{Light Intensity}
Natural light and how to calculate the intensity on a tilted surface.

% To convert the irradiance on a tilted surface we calculated before to the \ac{par}, there exists a simple conversion factor @reis2020.
% Meanwhile, when using \textit{artificial} lighting, \ac{ppfd} is used to describe the relevant radiation.
% \ac{par} and \ac{ppfd} quantify \textit{light spectrum} and \textit{light intensity} into one value.

\subsubsection{Time Aggregation}
When assessing the optimal lighting conditions for plant growth, several factors need to be illuminated.
Lol illuminated.
Light spectrum, Instantenous light intensity, Cumulative light amount and Photoperiod

To quantify the \textit{cumulative light amount} and \textit{photoperiod} for a whole day, we simply accumulate \ac{par} and \ac{ppfd} over this interval.
This is called \ac{dli}.

For artificial lighting, the spectrum will lie inside the photosynthetically active spectrum, since they are made specifically for plant cultivation.
And so the ppfd is taken directly.
Further optimization can be done by adjusting the red, green, blue ratios.
This is not taken into account however, since we will illuminate human inhabited areas.
Therefore, white light is chosen to not disturb the inhabitants of the building with irritating light colors.

LEDs are chosen because of their high efficiency and possibility to adjust the light spectrum granuarly.

Typical values for solar radiation and artificial illumination.



\subsection{Irrigation}
\label{sub:fund-cea-irr}
Water and Nutrients in \ac{cea} are mixed and delivered to the plants directly by a process known as fertigation.
For the most part the roots are taken care of directly, without the use of any soil.
Substrates such as rockwool or perlite provide alternatives but are no necessity.
This soilless method of cultivation is referred to as \textit{Hydroponics}.
Hydroponic systems use less water and enable greater plant densities than traditional agriculture.
They offer high consistency and a tight control on water and nutrient delivery.

Multiple different techniques like \ac{nft}, deep water culture and \textit{Aeroponics} have developed over the years for differing use cases.
Aeroponic systems are special, in that the roots of the plants are not submerged in water.
Instead, they are surrounded entirely by air and either sprayed or misted with fog.
% This reduces water usage even further by 65\% \textcolor{Blue}{https://en.wikipedia.org/wiki/Hydroponics#cite_note-:1-48}.
This relieves two of the main problems with hydroponics.
Disease and aeration.
In case of a single infected plant, the disease can be carried by the nutrient solution to the whole system without proper sterilization.
In aeroponics all roots are sprayed with fresh solution, therefore contamination does not spread easily.
Additionally, as the underground part of the plant does not perform photosynthesis but certainly needs oxygen for cellular respiration, water in hydroponics needs to be aerated.
This can obviously be dropped if the root zone is suspended in air already.
Because of this enhanced gas exchange, in theory a wider variety of plants can be cultivated compared to systems which submerge the roots in water.
% In theory a system like this can sustain any plant species.
% However, especially for aeroponics only a smaller variety of plant species have been demonstrated to grow reliably.
% However, because of the tighter control requirements so the roots do not dry out, this technique is not industry standard.
% However, because of the more severe consequences of a malfunction -- the roots dry out quickly and the plants die off -- this technique is not industry standard.
However, roots dry out quickly and plants die in case of a malfunction.
Therefore, this technique is not industry standard and generally has tighter requirements for control.

For this work we will employ an aeroponic system because of the lightweight nature and high flexibility.
The strong requirement for control will be alleviated by the use of separate units -- the plant panels -- compartmentalizing any damage potential.

\textcolor{Blue}{Explain \ac{ec} and pH.} 

% Aeroponics less water, less nutrient use.



\subsection{Atmosphere}
As elaborated before, optimizing photosynthesis -> chemical components.

Optimizing the atmosphere in \ac{cea} boils down to one procedure.
Enhancing photosynthesis.
There are two chemical inputs required to make this process happen.
% Additionally, there needs to be energy put in.
CO$_2$ and H$_2$O.

\textit{CO$_2$} is quite straight forward.
The availability to the plant can be enhanced by elevating concentration in the surrounding air.
This is not necessary of course, but is routinely done to increase yields in greenhouse settings.
Secondly the plant needs \textit{water}.
However only a small amount of water is actually used in metabolic processes such as photosynthesis.
About \SI{99}{\percent} of the H$_2$O is actually transpirated \textcolor{Blue}{needs ref} to continually move nutrients up from the roots.
This is historically modeled for crops by a process known as evapotranspiration.
Combining evaporation from the soil and transpiration of the plant body.
Since we are not dealing with soil, we only need to look at transpiration.
The characteristic concept capturing this process into a single value is \ac{vpd}.
% The most important concept needed to understand this, is \ac{vpd}.
\textit{\ac{vpd}} (\si{\kPa}) describes humidity and temperature of the air.
It is calculated by first computing the \ac{svp} (\si{\kPa}) for a given temperature $T$ (\si{\degreeCelsius}), 
$$
SVP = 0.611 e^{\frac{17.27 T}{T + 237.3}}
$$
and then using \ac{rh} (\si{\percent}) to obtain
$$
VPD = SVP \times (1-\frac{RH}{100}) \quad \text{@howell1995}.
$$
\textcolor{Blue}{VPD and SVP cursive everywhere or straight in the equation? What's the convention?}

High \ac{vpd} means dry air.
Too high and the plant will close its pores to limit water loss, restricting photosynthesis.
A low value suggests that the air is already saturated and transpiration is also impeded.
Typical values range from ... to ... and the ideal value depends on the crop and its growth state.

% There are several factors influencing the health and productivity of crop plants.
% We will be looking into two plant models and explain their inputs.
% Evapotranspiration and Yield.

% For air surrounding the plant, there are three main properties which play a role in its growth and water usage.
% Wind speed, \ac{vpd} and $\text{CO}_2$ concentration.
% \textit{Wind speed} is quite self-explanatory.
% \textit{\ac{vpd}} (\si{\kPa}) combines the humidity and temperature into a single value.
% It is calculated by first computing the \ac{svp} (\si{\kPa}) for a given temperature $T$ (\si{\degreeCelsius}), 
% $$
% SVP = 0.611 e^{\frac{17.27 T}{T + 237.3}}
% $$
%  and then taking \ac{rh} (\si{\percent}) to obtain
% $$
% VPD = SVP \times (1-\frac{RH}{100}) \quad \text{@howell1995}.
% $$
% \textcolor{Blue}{VPD and SVP cursive everywhere or straight in the equation? What's the convention?}
% \textit{$\text{CO}_2$} is next to water the main input for photosynthesis.
Our concept will not implement carbon dioxide enrichment, since the farm air will interface with humans in the building.
% In \ac{cea}, levels are often elevated to achieve higher yields.
% This is not implemented in our concept, since the farm air will interface with humans in the building.
\ac{cea} facilities also oftentimes spend significant resources to condition the air with \ac{hvac} systems.
Following the theme of minimizing energy consumption, passive air cooling is chosen.

Other notable qualities of the atmosphere include air temperature, air speed and humidity.
These foster mostly the health of the plant and will not have a significant impact on photosynthesis.
Temperature and humidity in this field is usually described as the single value of \ac{vpd}.

