%Chapter 2
\chapter{Fundamentals}
\label{chap:fundamentals}

\section{Thermodynamics}
\subsection{Heat Transfer}
To be able to understand the implementation of the simulation discussed later in \textcolor{blue}{needs ref}, we first need to introduce the basics of heat transfer. 
There are three different forms in which heat transfer can occur.
First Conduction, which describes heat moving through a solid material. This property can be characterised by the heat conductivity $k$ for the specific material.
The mathematical model equation will be introduced later with the heat capacity in \ref{sub:ther-props}.
Second we have Convection.
This is heat transfer which happens between the surface of a solid material and a fluid.
Last the transfer can occur via radiation.
As the name implies this is basically heat energy transmitted with electromagnetic rediation.
These will all be modeled seperately later.
\textcolor{blue}{How about heat transfer via mass transfer?}
\textcolor{blue}{some sources also have advection - what's that?}
\textcolor{blue}{Difference between sensible and latent heat flow.}

\subsection{Other important thermodynamic properties}
\label{sub:ther-props}
Explain Heat Capacity.



\section{Systems Modelling Approach}
\section{Agricultural and CEA Basics}
%
