%Chapter 2
\chapter{Fundamentals}
\label{chap:fundamentals}

\section{Thermodynamics}
\textcolor{Blue}{How do I cite best here? Whole section is based on} @cengel2003.

\subsection{Types of Heat}
Heat can be classified into two different forms.
There is sensible heat which directly causes a temperature change in a material.
And there is latent heat which is responsible for the phase change of a material.
During the phase change, there is no temperature change from heat added into or subtracted from the system.
% Water evaporating from a leave falls into this category.
Total heat transferred during a process is denoted by $Q$ and the rate at which this happens is signified with $\dot{Q}$ carrying the unit Watt.
This heat transfer rate $\dot{Q}$ is what we will look at next.

\subsection{Heat Transfer}
Heat transfer can fundamentally occur in three different forms.
Conduction, Convection and Radiation.

\textit{Conduction} refers to heat moving through a material.
It is characterized by the heat conductivity $k$ specific to the substance in question and can be modeled by Fourier's law
$$
\dot{Q}_{cond} = -k \frac{A}{L} \Delta T
$$
where $A$ is the area through which the conduction takes place, $L$ is the distance and $\Delta T$ is the temperature difference.
\textit{Convection} is heat transferred on the boundary between a solid and a fluid.
The characteristic value for this interaction is the convection heat transfer coefficient $h$ while the mathematical description is given by Newton's law of cooling
$$
\dot{Q}_{conv} = h A \Delta T
$$
with $A$ being again the area, and $\Delta T$ the temperature difference.
\textit{Radiation} describes heat transfer via electromagnetic waves.
Any material possessing a temperature greater absolute zero will emit some heat to its surroundings.
For a black body -- an idealized concept absorbing all incident radiation -- this heat flux density is given by the Stefan-Boltzmann Law.
For real materials the emissivity $\epsilon$ and the objects' surface area $A$ are taken into account to get
$$
\dot{Q}_{rad} = \epsilon \sigma A (T^4 - T_{surr}^4)
$$
where $T$ is the material temperature, $\sigma$ is the Stefan-Boltzmann constant and $T_{surr}$ describes the temperature of an idealized sphere infinitely far from the object.
Looking at incoming radiation, we have the characteristic value of absorptivity $\alpha$.
This is combined with the incident radiation $\dot{Q}_{inci}$ to obtain 
$$
\dot{Q}_{abso} = \alpha \dot{Q}_{\text{inci}} 
$$
for captured heat flux by a material.

Conveniently these physical relations are already modeled in the Modelica Standard Library and further built upon with the Buildings Library.
\textcolor{Blue}{Should I introduce here already the modelica models which implement this?}



\subsection{Other relevant thermodynamic properties}
\label{sub:ther-props}
Explain Heat Capacity.\\
Heat transfer via mass transfer.\\
\textcolor{Blue}{Shall I put references here for what the fundamentals are needed? Like latent heat to determine evaporation cooling, mass transfer for ventilating?}



% \section{Systems Modelling Approach}

\section{Agricultural and CEA Basics}
To understand \ac{cea}, we need to understand photosynthesis.

In the center of \ac{cea} stands the plant.
The environment is crafted to provide optimal conditions.

\subsection{Illumination}
For illumination a few factors are of importance.
Instant irradiance and time dependent factors like dli and dark period.

For irradiance there are two important factors which define instant irradiance.
Spectrum and irradiance power.
\subsubsection{Irradiance on a tilted surface}

\subsubsection{Solar spectrum and photosynthesis}
When assessing the optimal lighting conditions for plant growth, several factors need to be illuminated.
Lol illuminated.
Light spectrum, Instantenous light intensity, Cumulative light amount and Photoperiod

For quantifying \textit{spectrum} and \textit{instantenous intensity} we will introduce \ac{par} and \ac{ppfd}.
As discussed before we want to 
How do we quantify the solar irradiance 
Plants use solar radiation in the spectrum from 400 nm to 700 nm \textcolor{Blue}{needs ref} for photosynthesis.
This is only a portion of the actual solar radiation which is hitting earth.
For the \textit{natural} solar radiation in the context of plant growth, the concept of \ac{par} is most often used.
This describes the solar radiation which lies inside the aforementioned range for photosynthesis and can be calculated with a simple conversion factor @reis2020.
% To convert the irradiance on a tilted surface we calculated before to the \ac{par}, there exists a simple conversion factor @reis2020.
Meanwhile when using \textit{artificial} lighting, \ac{ppfd} is used to describe the relevant radiation.
\ac{par} and \ac{ppfd} quantify \textit{light spectrum} and \textit{instantenous intensity}.
They both carry the same unit and except for their natural or artificial origin, can be treated the same.

To quantify the \textit{cumulative light amount} and \textit{photoperiod} for a whole day, we simply accumulate \ac{par} and \ac{ppfd} over one day.
This is called \ac{dli}.

For artificial lighting, the spectrum will lie inside the photosynthetically active spectrum, since they are made specifically for plant cultivation.
And so the ppfd is taken directly.
Further optimization can be done by adjusting the red, green, blue ratios.
This is not taken into account however, since we will illuminate outside areas.
Therefore, white light is chosen to not disturb the inhabitants of the building with irritating light colors.

LEDs are chosen because of their high efficiency and possibility to adjust the light spectrum granuarly.

Typical values for solar radiation and artificial illumination.



\subsection{Irrigation}
Water and Nutrients in \ac{cea} are mixed and delivered to the plants directly by a process known as fertigation.
For the most part the roots are taken care of directly, without the use of any soil.
Substrates such as rockwool or perlite provide alternatives but are no necessity.
This soilless method of cultivation is referred to as \textit{hydroponics}.
Hydroponic systems use less water and enable greater plant densities than traditional agriculture.
They offer high consistency and a tight control on water and nutrient delivery.

Multiple different techniques like \ac{nft}, deep water culture and \textit{aeroponics} have developed over the years for differing use cases.
Aeroponic systems are special, in that the roots of the plants are not submerged in water.
Instead, they are surrounded entirely by air and either sprayed or misted with fog.
% This reduces water usage even further by 65\% \textcolor{Blue}{https://en.wikipedia.org/wiki/Hydroponics#cite_note-:1-48}.
This relieves two of the main problems with hydroponics.
Disease and aeration.
In case of a single infected plant, the disease can be carried by the nutrient solution to the whole system without proper sterilization.
In aeroponics all roots are sprayed with fresh solution, therefore contamination does not spread easily.
Additionally, as the underground part of the plant does not perform photosynthesis but certainly needs oxygen for cellular respiration, water in hydroponics needs to be aerated.
This can obviously be dropped if the root zone is suspended in air already.
Because of this enhanced gas exchange, in theory a wider variety of plants can be cultivated compared to systems which submerge the roots in water.
% In theory a system like this can sustain any plant species.
% However, especially for aeroponics only a smaller variety of plant species have been demonstrated to grow reliably.
% However, because of the tighter control requirements so the roots do not dry out, this technique is not industry standard.
% However, because of the more severe consequences of a malfunction -- the roots dry out quickly and the plants die off -- this technique is not industry standard.
However, roots dry out quickly and plants die in case of a malfunction.
Therefore, this technique is not industry standard and generally has tighter requirements for control.

For this work we will employ an aeroponic system because of the lightweight nature and high flexibility.
The strong requirement for control will be alleviated by the use of separate units -- the plant panels -- compartmentalizing any damage potential.

% Aeroponics less water, less nutrient use.



\subsection{Atmosphere}
As elaborated before, optimizing photosynthesis -> chemical components.

Optimizing the atmosphere in \ac{cea} boils down to one procedure.
Enhancing photosynthesis.
There are two chemical inputs required to make this process happen.
% Additionally, there needs to be energy put in.
CO$_2$ and H$_2$O.

\textit{CO$_2$} is quite straight forward.
The availability to the plant can be enhanced by elevating concentration in the surrounding air.
This is not necessary of course, but is routinely done to increase yields in greenhouse settings.
Secondly the plant needs \textit{water}.
However only a small amount of water is actually used in metabolic processes such as photosynthesis.
About \SI{99}{\percent} of the H$_2$O is actually transpirated \textcolor{Blue}{needs ref} to continually move nutrients up from the roots.
This is historically modeled for crops by a process known as evapotranspiration.
Combining evaporation from the soil and transpiration of the plant body.
Since we are not dealing with soil, we only need to look at transpiration.
The characteristic concept capturing this process into a single value is \ac{vpd}.
% The most important concept needed to understand this, is \ac{vpd}.
\textit{\ac{vpd}} (\si{\kPa}) describes humidity and temperature of the air.
It is calculated by first computing the \ac{svp} (\si{\kPa}) for a given temperature $T$ (\si{\degreeCelsius}), 
$$
SVP = 0.611 e^{\frac{17.27 T}{T + 237.3}}
$$
and then using \ac{rh} (\si{\percent}) to obtain
$$
VPD = SVP \times (1-\frac{RH}{100}) \quad \text{@howell1995}.
$$
\textcolor{Blue}{VPD and SVP cursive everywhere or straight in the equation? What's the convention?}

High \ac{vpd} means dry air.
Too high and the plant will close its pores to limit water loss, restricting photosynthesis.
A low value suggests that the air is already saturated and transpiration is also impeded.
Typical values range from ... to ... and the ideal value depends on the crop and its growth state.

% There are several factors influencing the health and productivity of crop plants.
% We will be looking into two plant models and explain their inputs.
% Evapotranspiration and Yield.

% For air surrounding the plant, there are three main properties which play a role in its growth and water usage.
% Wind speed, \ac{vpd} and $\text{CO}_2$ concentration.
% \textit{Wind speed} is quite self-explanatory.
% \textit{\ac{vpd}} (\si{\kPa}) combines the humidity and temperature into a single value.
% It is calculated by first computing the \ac{svp} (\si{\kPa}) for a given temperature $T$ (\si{\degreeCelsius}), 
% $$
% SVP = 0.611 e^{\frac{17.27 T}{T + 237.3}}
% $$
%  and then taking \ac{rh} (\si{\percent}) to obtain
% $$
% VPD = SVP \times (1-\frac{RH}{100}) \quad \text{@howell1995}.
% $$
% \textcolor{Blue}{VPD and SVP cursive everywhere or straight in the equation? What's the convention?}
% \textit{$\text{CO}_2$} is next to water the main input for photosynthesis.
Our concept will not implement carbon dioxide enrichment, since the farm air will interface with humans in the building.
% In \ac{cea}, levels are often elevated to achieve higher yields.
% This is not implemented in our concept, since the farm air will interface with humans in the building.
\ac{cea} facilities also oftentimes spend significant resources to condition the air with \ac{hvac} systems.
Following the theme of minimizing energy consumption, passive air cooling is chosen.

